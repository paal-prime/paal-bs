\chapter{Metrical TSP algorithms}


\section{Problem description}

\defproblem{ Metrical Travelling Salesman Problem}
{ Complete graph $G = (V,d)$, where $d:V \times V \to\mathbb{R}$ is a metric. }
{ What is the shortest cycle in $G$ which comes through every vertex exactly once? }

See link\footnote{\url{http://en.wikipedia.org/wiki/Travelling_salesman_problem}}
for detailed description.

\section{Christofides 1.5-aox}
Christofides algorithm\footnote{\url{http://en.wikipedia.org/wiki/Christofides_algorithm}} is designed to work with instances of TSP where edge
weights satisfy triangle inequality. Let $G=(V, w)$ be an instance of TSP,
the algorithm is consisted of four steps:
\begin{enumerate}
\item Create minimum spanning tree $T$ over $G$. \\
Here we have implemented $\Oh(n^2)$ Prim's algorithm\footnote{\url{http://en.wikipedia.org/wiki/Prim's_algorithm}}.
Although for Euclidean spaces there exists faster $\Oh(n\log(n))$ algorithm the overall gain
would be insignificant so we didn't use it.
\item Let $O$ be a set of odd degree vertices in $T$.
	  Find minimum perfect matching $M$ in complete graph over $O$. \\
For this part we have used \textit{Blossom V}\footnote{\url{http://pub.ist.ac.at/~vnk/software.html}}
implementation of Edmond's Blossom algorithm\footnote{\url{http://en.wikipedia.org/wiki/Blossom_algorithm}}.
Worst case running time of this part is $\Oh(|O|^5)$ but in average case it behaves very well.
\item Combine edges of $M$ and $T$ to form a multigraph $H$.
\item Find Eulerian circuit $E$ in $H$. ($\Oh(|H|)$).
\item Make $E$ Hamiltonian by shortcutting visited nodes. ($\Oh(|E|)$).
\end{enumerate}
Summing this up, our implementation in worst case needs $\Oh(|V|^5)$ time but generally
is fast in comparison to our other techniques.

\section{2opt Local Search}

In 2-opt\footnote{\url{http://en.wikipedia.org/wiki/2-opt}} local search for TSP, a single step
consists of reverting a segment of the current cycle. Intuitively, we find a point
at which the route crosses over itself and reorder it so that is doesn't.

Since in metrical TSP distance between a pair of points is the same for both directions,
computing fitness of the new route simplifies to replacing a pair of edges and replacing it with another
pair.

We've tested 3 different structures for maintaining the current local search solution:
\begin{itemize}
\item array of vertices in order as they appear on the cycle.
\item Reversible Segment List\footnote{\url{http://www.hars.us/Papers/revi.pdf}}
\item Augmented splay tree.
\end{itemize}

\begin{tabular}{c|cc}
& vertex read cost & reverse cost \\\hline
array & $\Oh(1)$ & $\Oh(|V|)$ \\
RSL & $\Oh(\sqrt{|V|})$ & $\Oh(\sqrt{|V|})$ amortized \\
splay & $\Oh(\log(|V|)$ amortized & $\Oh(\log(|V|))$ amortized
\end{tabular}

In our implementation segment to reverse is selected with uniform probability,
excluding the degenerated cases - set of cycle edges has to actually change.
Evaluating fitness of neighbouring solution drawn requires to read vertices
on 4 positions in the cycle. The number of fitness evaluations dominates
the number of the actual reverses in the long run due to the fact that it
becomes harder and harder to find a neighbour with better fitness as we approach
the local optimum. As a consequence we have empirically observed that out of
3 data structure mentioned, the array performed best on TSPLIB (see Benchmarks)
test cases.

\section{Monte Carlo Tree Search}
\section{Benchmarks}
To evaluate the algorithms quality we have used TSPLIB\footnote{\url{http://comopt.ifi.uni-heidelberg.de/software/TSPLIB95/}}
symmetrical instances.
\section{Results}


