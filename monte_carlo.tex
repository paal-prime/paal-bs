\chapter{Monte Carlo Tree Search framework design}

\section{Preliminaries}
Describing optimization problem in terms of finding a sequence of decisions
that result in optimal solution seems natural to humans. In the this chapter we
will describe related approach used widely in combinatorial optimization and
artificial intelligence. We assume reader to be familiar with the concept of
\emph{decision tree}
\footnote{\url{http://en.wikipedia.org/wiki/Decision_tree}}.

\section{Motivation}
\emph{Monte Carlo Tree Search} (MCTS) is a metaheuristic for finding
near-optimal decisions in the decision space by building search tree and
evaluating its nodes according to random simulations.
MCTS is an iterative method, samples from many iterations are combined together
so that the best (or rather the most promising) decision from the current state
can be chosen based on aggregated statistics.

To implement MCTS-based approximation for given problem two domain-specific
components must be defined: an evaluation function which gives linear ordering
of feasible solutions and an algorithm for enumerating all possible states
reachable from given one by making single decision.
Both of these parts are usually simple, but MCTS algorithm itself is not. Many
problems may be encountered when implementing the algorithm.
\begin{itemize}
  \item maintaining and traversing tree structure requires plenty of tedious
    and error prone code, memory utilization is a significant problem given
    number of simulations that can be performed using modern computers
  \item statistics gathering and utilization code cannot be tested in any reasonable way
    \begin{enumerate}
      \item correct implementation is, in most cases, not guaranteed to find optimal solution
      \item errors can cause slight regression or improvement depending on test instances
      \item numerical stability of statistics manipulation procedures is a
      problem that a few would expect from combinatorial optimization algorithm
    \end{enumerate}
  \item there is a number of components that are common or interchangeable
    amongst majority of MCTS method implementations, each of these parts is
    worth generalization due to virtually no coupling with approximated problem
    \begin{itemize}
      \item tree structure modifications and traversing
      \item time execution control
      \item statistics aggregation and optimal decision extraction
      \item efficient decision and state propagation
    \end{itemize}
\end{itemize}

Proposed framework is our attempt to address these issues.

\section{General description}
Nomenclature introduced in this section is adjacent to the one used in [TODO].
Let us recall the basic operation of MCTS algorithm, for more precise
description please refer to mentioned publication.

MCTS method finds single near-optimal decisions from the current state (which
will sometimes be called \emph{root state}) according to statistical
information gathered from a number of iterations. A single iteration can be
divided into three phases.
We start by choosing the most urgent node of a tree using \emph{tree policy}.
We will usually think about this process as making a walk from the root of a
tree to some node. Choosing appropriate tree policy allows us to maintain
balance between exploration and exploitation.
As a result of applying tree policy we may reach a leaf of the tree,
representing state which is not necessary terminal, as we don't want to store
entire tree in the memory. It is tree policy's responsibility to decide whether
to expand this leaf by creating a new child node for each state which can be
reached with single decision.
Starting from this node (state) we use \emph{default policy} to evaluate it, in
the simplest case default policy will make a random sequence of moves until
terminal state (which can be evaluated directly) is reached.
Result of an evaluation is propagated backwards (up to the root) and statistics
in each node are updated.

Note that after performing desired number of simulations one can choose the
most promising (according to aggregated statistics) decision and update the
current state. In order to find near-optimal solution one usually needs to find
a sequence of decisions that lead from arbitrary state to some terminal state
representing feasible solution. This is obviously trivial if one has an
algorithm that finds a single decision. We will discuss possible methods of
performing this task later in this chapter.

\section{Search tree format}
One can choose from two opposite approaches to represent search tree.

One approach is to identify each node of a tree with concrete state and connect
them with edges (identified with decisions) in such a way that children of a
given node are those states that can be obtained from parent node's state by
making single decision. It should be obvious that initial state -- the one from
which we want to make a decision at the moment -- is represented by the root of
the tree.

The other approach would be to store initial state in the root, but instead of
storing states one can store transitions between them -- possible decisions.

Both ways may seem identical conceptually, but they are quite different from
the algorithmic and technical points of view.
Imagine a problem where updating state after making single decision is an
expensive process, in such case the states-oriented implementation can improve
performance of simulation -- we would save the cost of applying decision to the
states placed near root node in search tree.
On the other hand computer representation of an entire state tends to be much
larger than representation of a single decision, in such situations second
approach can save us both time and memory.
We could not come up with reasonable problem and solving strategy based on
MCTS, where the first approach has any advantages over second one. Therefore we
provide framework for building algorithms using the second approach only.

There is a one (quite ugly) hack that can be done in order to implement search
tree structure similar to the first approach using our framework. Since state
description, decision representation and aggregating function that applies
decision to given state and produces new one are specified by user one can do
things as follows: state would stay unchanged, decision will be described not
by an incremental but as a full destination state and aggregation function
should discard source state and return destination one encoded in 'decision'.
If properly implemented by a user this can be used with our framework with
virtually no overhead. Therefore the chosen approach is at least as expressive
(when memory and complexity constraints are taken into account) as each of
proposed.

\section{Domain dependency}
Our goal was to limit a coupling between specific problem and our framework, we
have decided to restrict domain-specific part of a MCTS-based algorithm to two
components, namely State and Move. Referring to previously introduced
nomenclature, States coincide with abstract states (nodes in the tree) and
Moves -- with decisions (edges in the tree).
The Fitness is a separate concept describing result of and evaluation of a
state. We have assumed that Fitness is represented by linearly ordered set
which is isomorphic with a subset of double type. Except for theoretical
precision issues, we found no drawbacks of this simplifying decision for any
real life MCTS application.

% TODO
\section{Concepts}
% TODO update expand description
We have identified a few concrete components of MCTS algorithm, that are in our
opinion generically atomic.

Tree -- responsible for maintaining structure of a decision tree
represented by nodes connected via edges.
Note that the tree has no knowledge about traversing strategy (tree policy or
default policy), decision or state evaluation, domain-dependency etc. One can
think about the tree as a (very refined) iterator stub, which is being filled
and used by other parts of a framework (in most cases supplied by end
programmer).
Since nodes are part of the tree and aggregated statistics are conceptually
connected with tree policy we could not store them in nodes directly. This is
why next concept has been introduced.

Payload -- maintains statistics describing single state, that are used by tree
policy. Note that this concept may be domain-dependent due to possible
domain-dependency of tree policy as discussed later on. The policy is also the
only component that can access, modify and understand information enclosed in
payload. Node is only an owner of the payload object (and indirectly stores the
assignment between state and payload).

Explore (aka tree policy) -- coincides with tree policy described before.
All examples discussed and implemented during the course of designing this
framework reduced the problem of finding the most urgent node in entire tree to
finding locally the most urgent child to descend into and finding path to
chosen node from the root. Therefore tree policy provides a function that
given a node, a state reached in this node and its children returns child that
should be examined recursively. Alternative approach would be to pass (using
some really fancy encoding) entire tree structure to the tree policy
explicitly. By requesting local decisions we make this simple both conceptually
and technically yet described interface has enough expressive power. We have
never came up with nor found in literature a reasonable strategy that needs
more global information about search tree.
Building path from root node makes sense also from philosophical point of view,
since after evaluating chosen node we would like to update all nodes on the
path to the root.
Possible domain-dependency of the tree policy is allowed since local decision
is made based on entire information stored in the node and the state reached by
making decisions encoded in edges traversed during path building (node
selection).
This possibility is usually ignored since it has been proven that MCTS
algorithms are extremely efficient when implemented as metaheuristics, without
any knowledge about the problem other than default policy and some black-box
algorithm to generate states reachable from given one by making one decision.
In such case tree policy makes a decision based on aggregated numerical results
of previous samples only. Tree policy is also responsible for propagating
consequences of making another sample in the tree. Its update procedure is
issued for each node on the path from chosen node to the root (in this order).

Move -- as described in discussion of domain dependency, stored as an
additional information and used mainly to restore state associated with chosen
node. Has no important role and is completely dependent on State.

State -- main component that encodes specific problem in MCTS algorithm.
It is responsible for complete implementation of default policy, evaluation of
Fitness for given (terminal state), generating Moves that can be done from
current state and establishing connection between Move and itself.
Note that from the framework's point of view there is no assumption about how
default policy estimates state's Fitness, there is no need for the state to
apply random sequence of moves to itself in order to obtain terminal state
which can be evaluated directly, this technique is pretty common though (and
advised as a good starting point).
It is beneficial to walk through the process of choosing node by the Tree
(according to Explore component). We start in the root by copying initial
state and then we feed the copy with Moves found on edges traversed
according to decisions of the tree policy, therefore each time we need to
provide some component with state of current node we can answer this request
without storing state in each node. This means that State must provide a
procedure to modify itself based on provided Move.

\section{Possible modifications}
In every experimental implementation we have created to test our design, we
have observed that each component is monolithic. Splitting (always possible,
not always desirable) would make things slower and harder to implement or
understand.

The question is whether it is possible to merge some of presented components.
Since MCTS algorithms are widely used in games application we had some
intuition where to draw the line between domain dependent and independent
parts. The design is very flexible and as it was said one can move this
boundary making virtually everything domain-aware. It is obvious that State
and Move components must be separated even thought they are strongly connected
with each other, this separation must be visible from the Tree's point of view
for efficiency purposes.
Merging Payload with Tree (actually a node in the tree) would make most of the
code dependent on Explore component or memory inefficient and harder to modify
if node would store some predefined set of statistics.
Explore component is critical -- it is domain-independent in most cases and
there exist many implementations easily interchangeable between different
problems.

\section{Finding decision sequence}
Our framework covers the problem of finding optimal decision given initial
state. The natural extension to finding a feasible solution (which is hopefully
near-optimal) or equivalently a sequence of decisions, can be done by
iteratively finding and applying decision until terminal state (and therefore a
feasible solution) is reached.

The reason why we decided not to implement the additional loop should be
obvious. The simple loop approach is probably the worst one can come up with.

As the number of decisions made increases the reachable state space shrinks (in
most cases exponentially, otherwise our approximation is as slow as brute-force
algorithm) therefore it sounds reasonable to reduce the number of simulations
per one decision.

For some problems we can approximate size of the state space reachable from
current state and if it is smaller than some problem and implementation
dependent value it might be profitable to perform an exhaustive search instead
of random sampling. This typically increases accuracy of the approximation and
in some cases reduces time and space cost as we have no overhead on maintaining
structure and meta data kept in the search tree. This approach has been proven
to work in many game solving applications of MCTS and introduced an improvement
confirmed by our tests as described later.

\section{Framework supplied implementations}
% TODO

