\documentclass[final,hyperref={pdfpagelabels=false}]{beamer}
\usepackage{grffile}
\mode<presentation>{\usetheme{I6pd2}}
\usepackage[english]{babel}
\usepackage[utf8]{inputenc}
\usepackage{polski}
\usepackage{amsmath,amsthm, amssymb, latexsym}
%\usepackage{times}\usefonttheme{professionalfonts}  % obsolete
%\usefonttheme[onlymath]{serif}
\boldmath
\usepackage[orientation=portrait,size=a0,scale=1.4,debug]{beamerposter}
% change list indention level
% \setdefaultleftmargin{3em}{}{}{}{}{}


%\usepackage{snapshot} % will write a .dep file with all dependencies, allows for easy bundling

\usepackage{array,booktabs,tabularx}
\newcolumntype{Z}{>{\centering\arraybackslash}X} % centered tabularx columns
\newcommand{\pphantom}{\textcolor{ta3aluminium}} % phantom introduces a vertical space in p formatted table columns??!!

\listfiles

%%%%%%%%%%%%%%%%%%%%%%%%%%%%%%%%%%%%%%%%%%%%%%%%%%%%%%%%%%%%%%%%%%%%%%%%%%%%%%%%%%%%%%
\graphicspath{{figures/}}
 
\title{\huge PAAL: Practical Approximation Algorithm Library}
\author{Piotr Jaszkowski, Mateusz Machalica, Grzegorz Prusak and Łukasz Solak}
\institute[University of Warsaw]{The Faculty of Mathematics, Informatics and Mechanics, University of Warsaw, Warsaw, Poland}
\date[May 24, 2013]{May 24, 2013}

%%%%%%%%%%%%%%%%%%%%%%%%%%%%%%%%%%%%%%%%%%%%%%%%%%%%%%%%%%%%%%%%%%%%%%%%%%%%%%%%%%%%%%
\newlength{\columnheight}
\setlength{\columnheight}{105cm}


%%%%%%%%%%%%%%%%%%%%%%%%%%%%%%%%%%%%%%%%%%%%%%%%%%%%%%%%%%%%%%%%%%%%%%%%%%%%%%%%%%%%%%
\begin{document}
\begin{frame}

	\begin{columns}		
		\begin{column}{.96\textwidth}
			\vspace{1cm}
			\begin{center}
			\veryHuge Aproksymacje VS Metaheurystyki
			\end{center}
			\vspace{1cm}
		\end{column}
	\end{columns}

	\begin{columns}
		\begin{column}{.46\textwidth}
			\begin{block}{Aproksymacja}
				Aproksymacjami nazywamy algorytmy znajdujące
				rozwiązania bliskie optymalnemu o gwarantowanym
				współczynniku\\ przybliżenia.\\
				Czy coś takiego...
			\end{block}
		\end{column}
		\begin{column}{.46\textwidth}
			\begin{block}{Metaheurystyka}
				Metaheurystyki traktują definicję problemu jako ,,czarną skrzynkę'',
				która dostając propozycję rozwiązania mówi tylko na ile jest ono
				dobre (tj. problem definiuje funkcję celu do zoptymalizowania).
				Przykłady: local search,annealing, mcts, sieci neuronowe, etc.
			\end{block}
		\end{column}
	\end{columns}
	
	\begin{columns}		
		\begin{column}{.96\textwidth}
			\vspace{1cm}
			\begin{center}
			\veryHuge starcie nastąpi na następujących problemach
			\end{center}
			\vspace{1cm}

			\begin{block}{Metrical Traveling Salesman Problem}
				tyle\\
				tyle\\
				tyle\\
				tyle\\
				tyle\\
				tyle\\
				tyle\\
			\end{block}
			\begin{block}{Steiner Forest}
				lol
			\end{block}
			\begin{block}{Metrical Uncapacitated Facility Location}
				lashdfia wmleoif	
			\end{block}
		\end{column}
	\end{columns}
  \vskip1ex
  %\tiny\hfill\textcolor{ta2gray}{Created with \LaTeX \texttt{beamerposter}  \url{http://www-i6.informatik.rwth-aachen.de/~dreuw/latexbeamerposter.php}}
  \tiny\hfill{Created with \LaTeX \texttt{beamerposter}  \url{http://www-i6.informatik.rwth-aachen.de/~dreuw/latexbeamerposter.php} \hskip1em}
\end{frame}
\end{document}


%%%%%%%%%%%%%%%%%%%%%%%%%%%%%%%%%%%%%%%%%%%%%%%%%%%%%%%%%%%%%%%%%%%%%%%%%%%%%%%%%%%%%%%%%%%%%%%%%%%%
%%% Local Variables: 
%%% mode: latex
%%% TeX-PDF-mode: t
%%% End:
