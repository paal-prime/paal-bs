\chapter{Steiner Forest algorithms}

\section{Problem description}
% Vazirani (22 Steiner Forest, Problem 22.1)
Steiner Forest Problem (SFP)\\
Given an undirected graph $G = (V, E)$, a cost function on edges $c : E \rightarrow Q^+$, and a collection of disjoint subsets of V, $S_1, \dots S_k$, find a minimum cost subgraph in which each pair of vertices belonging to the same set $S_i$ is connected.
% Problem background?
% Prior results/papers?

\section{Primal-Dual Method}
Description of 2-approximation algorithm [Vazirani].

\section{Create and break cycle}
Given some solution to Steiner Forest Problem ,,Create and break cycle'' (CBC) local search tries to improve solution fitness by adding random edge such that cycle is created and then removing random/the most expensive edge from this cycle, it's obvious that obtained graph is a feasible solution. After described operation it might happen that some edges are no longer needed like for example edges adjacent to leafs that don't belong to $S_i$ for any $i \in \{1, \dots k\}$. Therefore pruning procedure is used to ensure that only essential edges remains which might lead to further fitness improvement.
% mozna napisac, ze w 1967 ktos wpadl na ten pomysl, ale chyba oprocz tego nic z tym nie zrobil + nie udalo sie znalezc pracy w internecie

\section{Minimum spanning tree of active vertices set}
,,MST of active vertices set'' local search starts with some feasible solution $F = (V', E')$ to SFP. Presented local search keeps set of vertices $A$ that at first equals $V'$. In each iteration local search randomly choose to remove vertex from $A$, add vertex to $A$, or do both, let's denote such changed set $A'$. Then in graph $G'$ induced by vertices of $A'$ minimum spaning tree for each connected component is found and pruned. If obtained forest is a feasible solution to SFP and it's fitness is better than any previously found, $A'$ becomes $A$ and best fitness found is updated.

\section{Monte Carlo Tree Search}
WIP.

\section{Benchmarks}
Comparison of algorithms results (including some variations) on different test cases (description of SteinLib + problem instances generated by us).

\section{Results}
Pros and cons of different approaches. Encountered problems.
