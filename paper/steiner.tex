\chapter{Steiner Forest algorithms}

\section{Problem description}
Steiner Forest Problem (SFP)\\
Given an undirected graph $G = (V, E)$, a cost function on edges $c : E \rightarrow Q^+$, and a collection of disjoint subsets of $V$, $S_1, \dots S_k$, find a minimum cost subgraph in which each pair of vertices belonging to the same set $S_i$ is connected.
For $k = 1$ the problem is known as Steiner Tree Problem (STP).

\section{Primal-Dual Method algorithm}
Local searches written in our framework were compared to a 2-approximation algorithm described by Vazirani [Vazirani]. As far as we know, it's the best approximation algorithm for a SFP known to date.

\section{Create and break cycle local search}
Given some solution to Steiner Forest Problem ,,Create and break cycle'' (CBC) local search changes it by randomly adding edge between solution's vertices. If added edge created a cycle then the heaviest edge from this cycle is removed. It's obvious that obtained graph is still a feasible solution. After described operation it might happen that some edges are no longer needed like for example edges adjacent to leafs that don't belong to $S_i$ for any $i \in \{1, \dots k\}$. Therefore pruning procedure is used to ensure that only essential edges remains which might lead to further fitness improvement.

\section{Minimum spanning tree of active vertices local search}
,,MST of active vertices'' (MSTAV) local search starts with some feasible solution of SFP. At each step a new solution is created by computing a pruned minimum spanning tree of $G'$ which is a subgraph of $G$ induced by some set of vertices $V'$ called active vertices. $V'$ is obtained from vertices belonging to the previous solution after randomly applying one of the following operations:
\begin{itemize}
\item add a random vertex that's not already in the set,
\item remove a random vertex from the set,
\item do both operations described above.
\end{itemize}

Note that it might happen that there is no feasible solution in $G'$. Therefore at the end of each step local search checks whether proposed solution is feasible and if it is not then it reverts any changes.

\section{Benchmarks}
Comparison of algorithms results (including some variations) on different test cases (description of SteinLib + problem instances generated by us).

\section{Results}
Pros and cons of different approaches. Encountered problems.
