\chapter{Metrical TSP algorithms}


\section{Problem description}

\defproblem{ Metrical Travelling Salesman Problem}
{ Complete graph $G = (V,d)$, where $d:V \times V \to\mathbb{R}$ is a metric. }
{ What is the shortest cycle in $G$ which comes through every vertex exactly once? }

See link\footnote{\url{http://en.wikipedia.org/wiki/Travelling_salesman_problem}}
for detailed description.

\section{Christofides 1.5-apx}
Christofides algorithm\footnote{\url{http://en.wikipedia.org/wiki/Christofides_algorithm}} is designed to work with instances of TSP whose edge
weights satisfy triangle inequality. Let $G=(V, w)$ be an instance of TSP,
the algorithm consists of four steps:
\begin{enumerate}
\item Create minimum spanning tree $T$ for graph $G$. \\
Here we have implemented $\Oh(n^2)$ Prim's algorithm\footnote{\url{http://en.wikipedia.org/wiki/Prim's_algorithm}}.
Although for Euclidean spaces there exists faster $\Oh(n\log(n))$ algorithm, the overall gain
would be insignificant so we didn't use it.
\item Let $O$ be a set of odd degree vertices in $T$.
	  Find minimum perfect matching $M$ in complete graph over $O$. \\
For this part we have used \emph{Blossom V}\footnote{\url{http://pub.ist.ac.at/~vnk/software.html}}
implementation of Edmond's Blossom algorithm\footnote{\url{http://en.wikipedia.org/wiki/Blossom_algorithm}}.
Worst case running time of this part is $\Oh(|O|^5)$ but in average case it behaves very well.
\item Combine edges of $M$ and $T$ to form a multigraph $H$.
\item Find Eulerian circuit $E$ in $H$. ($\Oh(|H|)$).
\item Make $E$ Hamiltonian by shortcutting visited nodes. ($\Oh(|E|)$).
\end{enumerate}
Summing this up, our implementation in worst case needs $\Oh(|V|^5)$ time but generally
is fast in comparison to our other techniques.

\section{2opt Local Search}

In 2-opt\footnote{\url{http://en.wikipedia.org/wiki/2-opt}} local search for TSP, a single step
consists of reverting a segment of the current cycle. Intuitively, we find a point
at which the route crosses over itself and reorder it so that is doesn't.

Since in metrical TSP distance between a pair of points is the same for both directions,
computing fitness of the new route simplifies to replacing a pair of edges and replacing it with another
pair.

We've tested 3 different structures for maintaining the current local search solution:
\begin{itemize}
\item array of vertices in order as they appear on the cycle.
\item Reversible Segment List\footnote{\url{http://www.hars.us/Papers/revi.pdf}}
\item Augmented splay tree.
\end{itemize}

\begin{tabular}{c|cc}
& vertex read cost & reverse cost \\\hline
array & $\Oh(1)$ & $\Oh(|V|)$ \\
RSL & $\Oh(\sqrt{|V|})$ & $\Oh(\sqrt{|V|})$ amortized \\
splay & $\Oh(\log(|V|)$ amortized & $\Oh(\log(|V|))$ amortized
\end{tabular}

In our implementation segment to reverse is selected with uniform probability,
excluding the degenerated cases -- set of cycle edges has to actually change.
Evaluating fitness of neighbouring solution drawn requires to read vertices
on 4 positions in the cycle. The number of fitness evaluations dominates
the number of the actual reverses in the long run due to the fact that it
becomes harder and harder to find a neighbour with better fitness as we approach
the local optimum. As a consequence we have empirically observed that out of
3 data structure mentioned, the array performed best on TSPLIB (see Benchmarks)
test cases.

\section{Monte Carlo Tree Search}
We will refer to the concepts and definitions introduced in MCTS design
chapter. Let us remind that the only domain-dependent concepts in our MCTS
framework are Move and State and the former is completely dependent on the
latter.

It is widely known from games solving applications of MCTS method that we
should choose State concretization which minimizes branch factor of resulting
search tree. For TSP problem we chose path building approach:
\begin{itemize}
  \item State is a path with well-defined end and start point (permutation of a
    subset of all vertices)
  \item initial State is an arbitrary chosen start vertex
  \item Move represents choosing from remaining vertices the next one to be
    added to the end of the path
  \item terminal State is reached when all vertices are in the path
  \item Fitness of terminal State is equal to cost of a Hamilton cycle, note
    that path defines permutation of vertices and distance function is a proper
    metric
  \item Fitness estimate is a fitness of terminal state reached after applying
    a sequence of random decisions
\end{itemize}

We have introduced two improvements to above schema. In order to reduce branch
factor of resulting tree we have limited number of vertices taken into
consideration when choosing next move. A good heuristics here is to choose from
no more than $K$ vertices closest to the path's end.

To improve accuracy of each sample we have decided to replace last $L$ random
moves with exhaustive search, therefore Fitness estimate is set to minimum of
fitnesses of all terminal states visited during this search. Typically $N$ is
very small, for implementation described above we have determined that the best
accuracy to performance ratio can be achieved when $N = 4$, it should be
obvious that $N$ depends on the State implementation rather than size of the
input graph. Note that this method works well with certain types of policies
and statistics aggregation methods as discussed in results overview.

\section{Benchmarks}
To evaluate the algorithms quality we have used
TSPLIB\footnote{\url{http://comopt.ifi.uni-heidelberg.de/software/TSPLIB95/}}
symmetrical instances.

\section{Results}


