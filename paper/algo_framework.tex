\chapter{Algorithms' results presentation design}
\section{Motivation}

The interest of the contemporary researchers in the field of algorithms 
focuses on problems to which no unambiguously good solution is known. As new heuristic algorithms solving the
particular issue are developed, there arises a need of comparison of their
quality. This meta problem is very generic in its nature and hard to handle,
taking into constideration the variety of approaches the specialists are using.
As a result it is usually troublesome to make a reliable statistics among programs
written by different people or even repeat someone else's empirical results. (not to mention
the fact that it is often hard to access the source code a published paper is
based on).

For the purpose of this paper we have developed our own framework
for homogenuous and repetitive transformation of the algorithm output to a
publishable result. Together with a concept capable of wrapping virtually any
type of algorithm, we've obtained a framework which should allow anyone to
repeat experiments written by other people in his own execution environment
and plug in his own solution to make a reliable comparison.

\section{Concepts \& design patterns}

\subsection{Table (design pattern)}

Table impersonates any type of visual report.
It is responsible for running an algorithm (see Algorithm concept), providing
it with a logger (see Logger concept), processing logs and collecting human
readable data. It should be prepared to handle some user defined class of
algorithms. Intuitively, every diagram, table, grid with results, etc. should
have its own specialized Table. Table is only a design pattern, since is too
general to enforce any specific interface on it.

\subsection{Algorithm (concept)}

Encapsulates an algorithm execution. It is responsible for
initialization of the specific algorithm (for example: obtaining seed, 
selecting problem instance, setting execution constants, selecting strategies,
etc.), running the algorithm itself, logging statistical data for Table
(or at least transfering down the logger) and collecting the results.

\subsection{Logger (concept)}

Logger is supposed to effectively harvest the relevant information
about the performance basing on the fitness (assumed to be ultimate measure).
The actual behaviour of the Logger is chosen by Table instance, however
Logger's interface (consisting only of log() method) is fixed,
since similarly to Random (see Local Search Framework) it is propagated downward
(i.e. it is intrusive).

\section{Framework supplied implementations}

\subsection{GridTable}

GridTable presents the final fitness of algorithms on a series of test cases.
It outputs a tex table, in which columns correspond to the test cases and
rows to the algorithms. It collects no data in the middle of the algorithm
execution (algorithms receive a VoidLogger which does nothing).

\subsection{FIDiagram}

FIDiagram (fitness/iteration diagram) registers fitness over time,
for a series of algorithms on the same test case.
It outputs a line chart (in tex tikz) presenting the relation fitness/iteration
for these algorithms. To collect data, FIDiagram provides algorithms with
ImprovementLogger, which simply registers (fitness,iteration) pair every time the fitness
improves. FIDiagram cooperates well with search() from the Local Search Framework.

