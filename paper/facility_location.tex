\chapter{Metrical Facility Location algorithms}

\section{Problem description}

Problem instance consists of a full bipartite graph $G = (F \cup C, F \times
C)$, where elements of $F$ are called \emph{facilities} and those of $C$ are
called \emph{cities}, $o : F \to \mathbb{R}$ being a cost function of opening
single facility and $c : F \times C \to \mathbb{R}$ being a cost function of
connecting facility with a city. The connection costs satisfy triangle
inequality.

The problem is to find a subset of facilities $O \subset F$ to be opened and an
assignment of cities to opened facilities $a : C \to O$ in such a way that the
total cost $\sum_{f \in O} o(f) + \sum_{c \in C} c(a(c), c)$ is minimized.

\section{Primal-dual schema 3-apx}
We have created efficient implementation of combinatoric approximation
algorithm for the problem as a reference point for other discussed methods. The
algorithm achieves a constant approximation factor of 3 and runs in $\Oh(|F||C|
\log(|F||C|))$ time. It operates in primal-dual fashion trying to find feasible
solution for dual problem with possibly the biggest cost. Detailed description
can be found in TODO. % TODO vasirani

We have introduced a slight modification to original algorithm due to Jain and
Vasirani. Once the set of opened facilities is determined as in original
algorithm we create optimal assignment in $\Oh(|F||C|)$ time by assigning each
city to the closest facility. The assignment presented in original paper is
useful for estimating approximation factor of the algorithm though.

\section{Local Search 3-apx}

Local search with the following properties
guarantees\footnote{\url{http://www.cs.ucla.edu/~awm/papers/lsearch.ps}}
3-apx at local optimum:
\begin{itemize}
\item Search space consists of all subsets of facilities.
\item Fitness function is the same as in problem statement.
\item Valid step is of one of the forms:
	\begin{itemize}
	\item insert one facility to the set
	\item delete one facility from the set
	\item swap one facility from the set with one from outside the set.
	\end{itemize}
\end{itemize}

We have implemented 2 variants of Walkers (see: Local Search Framework design)
for this algorithm:
\begin{itemize}
\item[LS1)] take a random valid step and calculate fitness: $\Oh(|F||C|)$ per iteration
\item[LS2)] find the fittest step among the valid ones: $\Oh(|F|(|F|+|C|))$ per iteration
\end{itemize}

In both cases, time of single iteration is proportional to the problem
instance size. It is therefore necessary to start the local search from
the solution relatively close (in the topology described) to the local
optimum. Intuitively the optimal solution will consist of a small number
of facilities (especially for test cases with uniformly distributed facilities and cities),
therefore the initial solution has been set to an empty set.

\section{Monte Carlo Tree Search}

We will refer to the concepts and definitions introduced in MCTS design
chapter. Let us remind that the only domain-dependent concepts in our MCTS
framework are Move and State and the former is completely dependent on the
latter.

Because MCTS involves selecting decisions step by step we decided to transform our
standard Facility Location problem into Online Facility Location problem.
\begin{itemize}
\item State is a vector of already opened facilities and an ordering in which
next facilities will be processed.
\item Initial State is empty vector (opened facilities) and a permutation
of $1,2,\dots, |F|$ (ordering).
\item Move represent a choice if facility should be opened in given point or not.
\item Terminal state is when all facilities have been processed (each one is
considered only once).
\item Fitness of terminal State is equal to sum of openning costs and connection costs.
\item Fitness estimate is calculated using Adam Meyerson's algorithm\footnote{\url{http://www.cs.ucla.edu/~awm/papers/ofl.pdf}}
which is constant competitive for randomized input.
\end{itemize}

\section{Random Search}

As an additional benchmark for evaluating algorithms' performance, we've implemented
random search, which simply samples solution space at random. Solutions are drawn as follows:
\begin{itemize}
\item draw with uniform distribution the solution size $n \in \{1..|F|\}$.
\item draw $n$ times a facility with uniform distribution.
\item solution is the set of facilities from the previous point.
\end{itemize}

We've empirically shown for the benchmark problem instances, that the near optimal
solutions are small sets. This drawing algorithm is flexible enough to take it
into consideration.

\section{Benchmarks}
To evaluate the algorithms quality we have used UflLib\footnote{\url{http://www.mpi-inf.mpg.de/departments/d1/projects/benchmarks/UflLib/Euklid.html}}
and our "clustered tests". The need for "clustered tests" raised from observation that
uniform distribution is not natural for many real-world situation like city placements
e.g. normally cities are "clustered" around natural resources.

\section{Results}

\begin{figure}[ht]
  \begin{tabular}[ht]{|l||c|c|c|c|c|c|H}
\cline{1-7}
 & 2511EuclS & 1811EuclS & 1211EuclS & 111EuclS & 1911EuclS & 2711EuclS & \\ \cline{1-7}\cline{1-7} 
optimum &99195 & 100189 & 98528 & 96116 & 98617 & 93845 & \\ \cline{1-7}
local search &1.0023 & 1.0165 & 1 & 1.01472 & 1.00671 & 1.00498 & \\ \cline{1-7}
ls fast &1.00605 & 1 & 1 & 1.00744 & 1.01128 & 1 & \\ \cline{1-7}
3 apx &1.19845 & 1.10865 & 1.23954 & 1.09184 & 1.10833 & 1.13838 & \\ \cline{1-7}
random &1.13476 & 1.12494 & 1.12943 & 1.1231 & 1.12784 & 1.11763 & \\ \cline{1-7}
\end{tabular}
\end{figure}
