\chapter{Local Search framework design}

\section{Motivation}

Local search\footnote{\url{http://en.wikipedia.org/wiki/Local_search_(optimization)}} is a metaheuristic for solving optimization problems.
It is especially useful when the fitness function's local optima have value close to the global one.
Once a decent topology on the solution space is defined, the algorithm itself is very easy to implement:
you just make a walk in the solution space, choosing every step so that the next solution you "stand in" has
a better fitness than the previous one. The local search implementation has usually just a few dozens of lines and
yet it can generate a series of problem, which usually stay unnoticed:
\begin{itemize}
\item it is hard to test a local search; local searches are prone to bugs:
	\begin{itemize}
	\item correct implementation does not guarantee that the result will be optimal
	\item there is usually no objectively good benchmark instances for problems to which local search is applied
	\item quality of the solution usually strongly depends on the amount of available computational power
	\item nobody bothers to write a decent testing framework (since every local search implementation is different).
		As a result tests are runned by hand, which makes testing prone to human errors (we can easily lose repetitiveness of the results).
	\end{itemize}
\item some parts of code are rewritten in every implementations:
	\begin{itemize}
	\item execution time control
	\item main local search loop
	\item fitness monitoring
	\item step decision making strategy
	\end{itemize}
	Every single of them takes not much code which is hard to generalize anyway.
	However, it is easy to make a bug in these places, which can stay unnoticed for a long time.
\end{itemize}

We've made an attempt to implement a C++ local search 
framework addressing these issues.

\section{General description}

Local Search Framework is supposed to automatize the process of writing local searches with no execution overhead due to the framework code.
Framework should be responsible for making consistent decisions about the issues that are not influencing the algorithm itself
(for example results logging, repetitive testing, limiting execution time) and allow to avoid rewriting repetitive code.
As the concept of local searches is really simple, our design has to be easily comprehensible and super intuitive.
It definitely shouldn't force user to bend/hack the solution to fit the framework.

We use templates and concepts to avoid any overhead in the execution time and allow maximum flexibility.

[main function] search()

Outer interface of the framework consists of a single template function search().
Its code is explicitly stated in the design - user HAS TO KNOW this piece of code before using the framework.
The order of actions performed is vital for utilization of the framework.
Therefore search() has to be simple and readable for an average user.

fitness - we assumed that fitness of the solution can be efficiently calculated and represented as a floating point number.
We believe that imposing the fitness type across the framework is a useful simplification and prevents any type conversion problems in this context.
In the current version fitness type is set to double which is disputable but convenient.

On the top level of the framework, ie. search() and concepts it uses we don't know the nature of the problem.
It is due to the fact that we wanted to extract problem independent components.

[concept] Walker - only component which knows the nature of the problem.
	It is not divided on this framework level, since if more components would know about the problem it would create cross dependencies.
	In other words, such situation would inevitably make search() to transfer problem specific data between them.
	Also many custom/intrusive optimization can be made at this point, so we believe that this is definitely a point at which we should allow user to plug in
	his own code.

[concept] PRNG (pseudorandom number generator) has to be shared, since only 1 seed is expected to be provided and using multiple PRNGs can cause conditional probabilities.

[concept] ProgressCtrl (progress controller) - is usually implemented as a time/iteration limit or sufficiency threshold.
None of these are problem dependent, hence this component can be effectively extracted from the design.

[concept] StepCtrl (step controller) - component deciding whether to perform a step - represents the metaheuristic used.
annealing and hill climb have easy/trivial implementations for this concept.

\section{Main function}

\begin{lstlisting}
template<typename Walker, typename StepCtrl, typename ProgressCtrl,
	typename Random, typename Logger>
void search(Walker &walker, StepCtrl &step_ctrl, ProgressCtrl &progress_ctrl,
	Random &random, Logger &logger)
{
	while(1)
	{
		double current_fitness = walker.current_fitness();
		logger.log(current_fitness);
		double progress = progress_ctrl.progress(current_fitness);
		if(progress>=1) break;
		walker.prepare_step(progress,random);
		if(step_ctrl.step_decision(current_fitness,walker.next_fitness(),
			progress,random)) walker.make_step();
	}
}
\end{lstlisting}

\section{Concepts}

\subsection{Walker}

\begin{lstlisting}
concept Walker
{
	/*
		Walker is the only object which actually knows the nature of the problem.
		It is responsible for:
			- maintaining the current solution
			- preparing the proposition of the new step
			- making the proposed step

		It has to contain an initial solution before calling search().
		Better solutions have lower fitness.

		Input:
			progress \in [0,1)
			Random
		Output:
			current_fitness
			next_fitness
	*/

	template<typename Random> void prepare_step(
		double progress, Random &random);
	/* 
		Prepares a proposition of a single step (it doesn't change the current
		solution). Let next solution be the solution after making step from
		the current solution. Fitness of the next solution should be returned
		by next_fitness(). search() provides that prepare_step() will be called
		in each iteration exactly once.
	*/

	void make_step();
	/*
		Performs the step prepared by the last execution of prepare_step(),
		i.e. the current solution shall change to the next solution.
	*/
	
	double current_fitness();
	/*
		Returns fitness of the current solution.
		It changes only during execution of make_step().
	*/

	double next_fitness();
	/*
		Returns fitness of the next solution (see prepare_step()).
		Before the first call of prepare_step(), behaviour of next_fitness()
		is undefined.
	*/
};
\end{lstlisting}

\subsection{StepCtrl}
\begin{lstlisting}
concept StepCtrl
{
	/*
		StepCtrl (Step Controller) represents the metaheuristic chosen to
		solve the problem. It is responsible for making decision whether to
		make the step proposed by Walker.
		
		Input:
			current_fitness
			next_fitness
			progress \in [0,1)
			Random
		Output:
			step_decision
	*/

	template<typename Random> bool step_decision(double current_fitness,
		double next_fitness double progress, Random &random);
	/*
		Returns true iff step proposed by the Walker should be made.
		
		examples:	
			In annealing, the decision would be made according to the
				fitness delta and the temperature schedule.
			In hill climb, the decision would be solely based on fitness delta.
			{ What should be done for tabu search? Do we really care? }
	*/
};
\end{lstlisting}

\subsection{ProgressCtrl}

\begin{lstlisting}
concept ProgressCtrl
{
	/*
		ProgressCtrl (Progress Controller) controls the execution time of the
		local search. It is responsible for estimation of the ratio: iterations
		passed/iterations available.

		Input:
			current_fitness
		Output:
			progress \in [0;inf)
	*/

	double progress(double current_fitness);
	/*
		Returns non-negative value estimating the ratio described above. search()
		provides that progress() will be called exatly once before every
		iteration. Returning value >=1 will cause search() to exit without
		performing the next iteration.

		examples:
			[iteration limit]
				(iterations passed)/(predetermined number of iterations available)
			[time limit] (time passed)/(predetermined time available)
			[sufficiency treshold] (acceptable fitness)/best_fitness
			[convergence treshold]
				max(0,1+\epsilon-c*((best fitness x iterations ago)-best_fitness))
	*/
};
\end{lstlisting}

\subsection{Random}

\begin{lstlisting}
concept Random [c++11 RNG concept];
/*
	Generates random integer values.
	It is explicitly shared between components of the search() for the following reasons:
	- we assume that we receive only one seed from the outside
	- using many generators at once (especially with the same seed) may create undesired conditional probabilities
*/
\end{lstlisting}

\subsection{Logger}

TODO: move it to algorithm framework

\begin{lstlisting}
concept Logger
{
	/*
		It logs fitness over algorithm execution time/iterations.
		{ To what extent does it overlap with the ProgressCtrl? Should it take progress argument? }

		Input:
				current_fitness
	*/

	void log(double current_fitness);
	/*
		It logs current fitness.
		search() provides that log() will be called exactly once before every iteration once after the last one.
	*/
};
\end{lstlisting}

\section{Features for implemening Walker}
[TO BE REMOVED]

We have observed that sometimes it is possible to enhance the local search
by making the neighbourhood size (from which we choose the step) progress dependent.
In the euclidian TSP example, assuming that our step consists of reversing a segment of the cycle,
it occurs to be profitable to decrease the expected length of that segment along the time
(intuitively it is equivalent to untying smaller and smaller "knots" in our solution).
We suspect that it can be generalized, therefore a StepSize concept has been introduced.
We are going to implement a few potentially useful variants of StepSize,
although the concept itself is not a part of the local search framework.

\begin{lstlisting}
concept StepSize
{
	/*
		Hints a length of the step (neighbourhood ball radius) to generate.

		Input:
				Random
				progress \in [0;1)
		Output:
				step_size
	*/

	uint32_t hint(double progress, Random &random); { Why uint32_t? }
	/*
		Returns the length of the step generated according to some progress dependent distribution.

		examples:
				TODO
	*/
};
\end{lstlisting}

\section{Reasoning}

We have considered approaches alternative to the proposed design, they are
worth studying in order to identify possible problems and why current design
solves them in our opinion.

\subsection{Abstract representation}
It is helpful to express structure of local search framework in terms of a
graph where each node represents a separated concept and an edge $A \to B$
exists iff concept $A$ uses (e.g. calls, owns, reads data) concept $B$. First
assumption was that resulting graph is compact, local search algorithm provided
by the framework should be seen as one phase from the outside (e.g. invocation
of one function, which takes input and returns result), so every concept in
different connected component than the algorithm itself is useless. We will
refer to described graph as the \emph{component graph}.

\subsection{The way to the tree}
We have eliminated all designs with cycles in component graph, the most
important reason for this was that we could not find how we can benefit from
allowing two way communication between components. Furthermore all concepts
need to be initialized either by user or search procedure itself if they are
completely algorithm independent (like statistics gathering) and every cycle
makes initialization really complicated, there is no arbitrary choice of
initialization order we could come up with in this case.

Therefore we were free to choose either general DAG or a tree (we were aiming
for the latter). To reduce both of them to the second one we can run
depth-first search from algorithm's entry point (the search() procedure), which
becomes a root of a depth-first spanning tree. The only difference between the
graph being a general DAG and a tree is manifested by the presence of non-tree
edges. Our graph is acyclic, so we will not find back edges.

Please note that presence of back edges would mean that execution flow can be
returned to caller for a moment and then returned back to the callee
automatically - this can obviously be replaced by making communication
passive, caller provides all data needed do execute call (query) and expects
response, every query is initiated by caller and completed by callee using
provided data.

\subsection{Forward edges elimination}
Any forward edge means that we are skipping many abstraction layers with an
invocation. Obviously this can be avoided by passing a call through all layers
that are bypassed by a forward edge. We replace forward edge with composition
of facades. Modern C++ compilers can easily inline such constructions, so that
the forward edge will not be present in design and C++ code, but will be
generated by compiler for performance reasons.

Moreover every forward edge means that at least two different concepts need to
know (the same) type of called concept, we think it is less convenient during
initialization, however in this specific case a concept can inherit type from
its predecessor in the spanning tree (as long as the former is not initialized
by a user).

We have made an experiment with changing parts of an algorithm (namely
underlying data structure) during algorithm execution. Obviously allowing
forward edge in this case (calling specific data structure directly, skipping
its owner) might invalidate algorithm. The real problem here is a stupid user
who gave an ownership to the object being descendant of a caller while allowing
forward edges in the component graph. Removing forward edge as described above
forces every call to pass through data structure owner, which is less
error-prone and simpler.

We further assume that there is no forward edges in the graph.

\subsection{Cross edges discussion}
Consider the case where one concept $C$ is a child of two (or more) concepts
$A$ and $B$ (assume $B \to C$ is a cross edge). Let $P$ be the lowest common
ancestor of both $A$ and $B$ in depth-first spanning tree of the component
graph. Described situation breaks composability of algorithm's design but one
can easily come up with an example of a local search algorithm where such
design is desired, we will discuss it later, for now let us assume that we do
need communication between components as described above.

If we disallow cross edges in our design we have to provide different mechanism
of communication between $B$ and $C$. As the graph without cross edges is a
tree we have to pass data back (query result) from $C$ to $P$ and provide $B$
with it in separate call. The obvious problem here is that data format must be
know to all concepts on paths from $B$ and $C$ do $P$ (including $P$) and data
format type can (in most cases will) be domain dependant - therefore we are
raising domain boundary up to lowest common ancestor of components that were
connected by cross edge. This means that protocol common to $B$ and $C$ (which
may change) must be propagated during initialization down in the tree starting
from $P$.

The obvious benefits from tree-like component graph is simple. In top-down
initialization, we can assume that during construction of algorithm child being
initialized is given an interface to obtain initialization data from parent
according to its needs or this data is passed explicitly. Furthermore there is
no problem with assigning an ownership of a concept. Tree design makes
synchronization model extremely easy to implement in case one would like
concepts to operate concurrently.

There are more problems with cross edges. Initialization cannot be done
top-down in described automatic fashion, ownership assignment just like the
rest of the initialization procedure needs to be done by the user as hidden
dependencies are not obvious to any automatic model. It is pretty obvious that
the less constraints we put on algorithm's design the less we know and the more
must be done manually.

It turned out that in the case of local search framework both performance
considerations and domain-dependency limiting required nearly the same
transformations of components graph, which resulted in the graph becoming a
tree. Cross edges were eliminated by encoding entire problem specific
components in the Walker concept. Topologies of solution spaces implied by
problems discussed when designing this framework vary considerably, we were
unable to split this concept leaving complexity of single step unchanged for
studied examples.

\section{Framework supplied implementations}
