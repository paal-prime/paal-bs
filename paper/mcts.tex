\chapter{Monte Carlo Tree Search framework design}

\section{Preliminaries}
Describing optimisation problem in terms of finding a sequence of decisions
that result in optimal solution seems natural to humans. In this chapter we
will describe related approach used widely in combinatorial optimisation and
artificial intelligence. We assume reader to be familiar with the concept of
\emph{decision tree}
\footnote{\url{http://en.wikipedia.org/wiki/Decision_tree}}.

\section{Motivation}
\emph{Monte Carlo Tree Search} (MCTS) is a metaheuristic for finding
near-optimal decisions in the decision space by building the search tree and
evaluating its nodes according to random simulations.
MCTS is an iterative method - samples from many iterations are combined together
so that the best (or rather the most promising) decision from the current state
can be chosen based on statistics aggregated.

To implement MCTS-based approximation for given problem two domain-specific
components must be defined: an evaluation function which gives a linear ordering
of feasible solutions and an algorithm for enumerating all possible states
reachable from given one by making a single decision.
Both of these parts are usually simple, but MCTS algorithm itself is not. Many
problems may be encountered when implementing the algorithm:
\begin{itemize}
  \item maintaining and traversing tree structure requires plenty of tedious
    and error prone code; also memory utilisation is a significant problem given the
    number of simulations that can be performed using modern computers
  \item statistics gathering and utilisation code cannot be tested in any reasonable way:
    \begin{enumerate}
      \item correct implementation is, in most cases, not guaranteed to find the optimal solution
      \item errors can cause slight regression or improvement depending on test instances
      \item numerical stability of statistics manipulation procedures is a
      problem that a few would expect from combinatorial optimisation algorithm
    \end{enumerate}
  \item there is a number of components that are common or interchangeable
    among the majority of MCTS method implementations, each of these parts is
    worth generalisation due to virtually no coupling with the approximated problem:
    \begin{itemize}
      \item tree structure modifications and traversing
      \item time execution control
      \item statistics aggregation and optimal decision extraction
      \item efficient decision making and state propagation
    \end{itemize}
\end{itemize}

Proposed framework is our attempt to address these issues.

\section{General description}
Nomenclature introduced in this section is similar to the one used in
\cite{MCTSsurvey}.  Let us recall the basic operation of MCTS algorithm. For
more precise description please refer to the publication mentioned.

MCTS method finds single near-optimal decisions from the current state (which
will be sometimes called a \emph{root state}) according to statistical
information gathered from a number of iterations. A single iteration can be
divided into three phases.
We start by choosing the most urgent node of a tree using \emph{tree policy}.
We will usually think about this process as making a walk from the root of a
tree to some node. Choosing appropriate tree policy allows us to maintain
balance between exploration and exploitation.
As a result of applying tree policy we may reach a leaf of the tree,
representing state which is not necessarily terminal, as we don't want to store
entire tree in the memory. It is tree policy's responsibility to decide whether
to expand this leaf by creating a new child node for each state which can be
reached with a single decision.
Starting from this node (state) we use \emph{default policy} to evaluate it.
In the simplest case default policy will make a random sequence of moves until
terminal state (which can be evaluated directly) is reached.
The result of an evaluation is propagated backwards (up to the root) and statistics
in each node are updated.

Note that after performing the desired number of simulations one can choose the
most promising (according to aggregated statistics) decision and update the
current state. In order to find near-optimal solution one usually needs to find
a sequence of decisions that lead from arbitrary state to some terminal state
representing feasible solution. This is obviously trivial if one has an
algorithm that finds a single decision. We will discuss possible methods of
performing this task later in this chapter.

\section{Search tree format}
One can choose from two opposite approaches to represent a search tree.

One approach is to identify each node of a tree with concrete state and connect
them with edges (identified with decisions) in such a way that children of a
given node are those states that can be obtained from the parent node's state by
making a single decision. It should be obvious that initial state -- the one from
which we want to make a decision at the moment -- is represented by the root of
the tree.

The other approach would be to store initial state in the root, but instead of
storing states in all other nodes one can just store transitions between them -- possible decisions.

Both ways may seem identical conceptually, but they are quite different from
the algorithmic and technical points of view.
Imagine a problem where updating state after making a single decision is an
expensive process, in such case the states-oriented implementation can improve
performance of simulation -- we would save the cost of applying decision to the
states placed near root node in the search tree.
On the other hand representation of an entire state tends to be much
larger than representation of a single decision and in such situations second
approach can save us both time and memory.
We were unable to come up with a reasonable problem and a solving strategy based on
MCTS for which the first approach would have any advantages over the second one.
Therefore we provide the framework for building algorithms using the second approach only.

There is a (quite ugly) hack that can be done in order to implement search
tree structure similar to the first approach using our framework. Since both the
state description, the decision representation and the aggregating function (that applies
decision to given state and produces new one) are specified by the user, one can
organize things as follows: state would stay unchanged, decision will be described not
by an incremental but by a full destination state, and aggregation function
should discard source state and return the destination encoded in the 'decision'.
If properly implemented by a user this can be used within our framework with
virtually no overhead. Therefore the chosen approach is at least as expressive
(when memory and complexity constraints are taken into account) as each of the
proposed ones.

\section{Domain dependency}
Our goal was to limit coupling between specific problem and our framework.
We have decided to restrict domain-specific part of a MCTS-based algorithm to two
components, namely State and Move. Referring to previously introduced
nomenclature, States coincide with abstract states (nodes in the tree) and
Moves -- with decisions (edges in the tree).
The Fitness is a separate concept describing the result of and an evaluation of a
state. We have assumed that Fitness is represented by a linearly ordered set
which is isomorphic with a subset of double type. Except for theoretical
precision issues, we found no drawbacks of this simplifying decision for any
real life MCTS application.

\section{Concepts}
We have identified a few components of MCTS algorithm, that are in our
opinion generically atomic.

\subsection{Payload}
Tree node contains 2 unrelated informations: the problem specific decision (Move)
and the policy specific statistics (Payload).
Payload is a passive structure which is being defined by the Policy
and the Policy is the only component that can access,
modify and understand information enclosed in Payload.

\subsection{Policy}
The Policy concept coincides with the tree policy described before.
All examples discussed and implemented during the course of designing this
framework reduced the problem of finding the most urgent node in entire tree to
finding locally the most urgent child to descend into and in consequence
finding a path to a chosen node from the root. Therefore tree policy provides a
function that given a node, a state represented by this node and its children
returns child that should be examined recursively. An alternative approach would
involve passing the entire tree structure to the decisive function. By requesting
local decisions we make this simple both conceptually and technically yet
described interface has enough expressive power. We have never came up with nor
found in the literature a reasonable strategy that requires global information
about the search tree.
Building path from the root node enables us to limit the effects of performing
a single simulation to a reasonable number of nodes, since after evaluating chosen
node we update all nodes on the path to the root and only them.

Possible domain-dependency of the tree policy is theoretically allowed since
the local decision is based on the entire information stored in the node and the
state reached by making decisions encoded in edges traversed during node
selection.
This possibility is usually ignored since it has been proven that MCTS
algorithms are extremely efficient when implemented as metaheuristics, without
any knowledge about the problem other than default policy and some black-box
algorithm to generate states reachable from given one by making a single
decision. In such case tree policy makes a decision based on aggregated results
of previous simulations only.
This is also an approach that we have chosen for our framework.

Policy is also responsible for propagating consequences of making a single
sample in the tree. Its update procedure is issued for each node on the path
from the chosen node to the root (in this order).

Once algorithm reaches the most urgent node it has to decide whether to expand
it -- create child nodes for each state reachable by making a single decision.
The Policy concept provides a predicate which answers this question based on
reached node and its level in the tree (number of nodes on the path to tree's
root).

\subsection{Move}
Move is an increment representing a single decision. It is stored in the search
tree instead of full state and incrementally applied to the state when making
simulations. It has no important role and is completely dependent on the State.

\subsection{State}
State is a main component that encodes specific problem in MCTS algorithm.  It
is responsible for full implementation of the default policy, evaluation of Fitness
for a given state, generating Moves that can be done from the current state and
establishing connection between Move and itself.

Note that from the framework's point of view there is no assumption about how
default policy estimates state's Fitness. As a result there is no need for the state to
apply a random sequence of moves to itself in order to obtain terminal state
which can be evaluated directly. This technique is pretty common though (and
recommended as a good starting point).

It is beneficial to walk through the process of choosing the node by the Tree
(according to Policy component). We start at the root by copying initial state
and then we feed the copy with Moves found on the edges traversed according to
the decisions of the tree policy. Therefore each time we need to provide some
component with the state of current node we can answer this request without
storing state in each node. This also means that State must provide a procedure
to modify itself basing on provided Move.

\section{Possible modifications}
In every experimental implementation we have created to test our design, we
have observed that each component is monolithic. Splitting (always possible,
not always desirable) would make things slower and harder to implement or
understand.

The question is whether it is possible to merge some of presented components.
Since MCTS algorithms are widely used in games application we had some
intuition where to draw the line between domain dependent and independent
parts. The design is very flexible and as it was said one can move this
boundary making virtually everything domain-aware. It is obvious that State
and Move components must be separated even thought they are strongly connected
with each other. This separation must be visible from the Tree's point of view
for efficiency purposes.
Merging Payload with Tree (actually a node in the tree) would make most of the
code Policy dependent or memory inefficient and harder to modify if the node
would store just some predefined set of statistics.
Policy component is critical -- it is domain-independent in most cases and
there exist many implementations easily interchangeable between different
problems.

\section{Finding decision sequence}
Our framework covers the problem of finding optimal decision given initial
state. The process of finding a feasible solution (which is hopefully
near-optimal) can be decomposed into iterations consiting of finding and applying 
a single decision until the terminal state
(and therefore a feasible solution) is reached.

The reason why we decided not to implement this external loop should be
obvious. The simple loop approach is probably the worst one can come up with.

As the number of decisions made increases, the space of reachable states shrinks
(in most cases exponentially, otherwise our approximation is as slow as brute-force
algorithm) therefore it sounds reasonable to reduce the number of simulations
per one decision over time.

For some problems we can estimate the size of the space of states reachable from
the current state and if it is smaller than some problem and implementation
specific value, it might be profitable to perform an exhaustive search instead
of random sampling. This typically increases accuracy of the approximation and
in some cases reduces the time and space cost as we have no overhead on maintaining
the structure and meta data kept in the search tree. This approach has been proven
to work in many game solving applications of MCTS and introduced an improvement
confirmed by our tests.

\section{Framework supplied implementations}

Together with MCTS framework we provide several implementations of concepts
which are domain independent and interchangeable between problems. Examples
include tree policies widely used in games solving MCTS algorithms.

In all cases expand decision and Fitness estimation are independent one from
another. To avoid writing all possible combinations we created components which
implement one or the other part of Policy's functionality and we've put them
together using inheritance.

\subsection{Expansion strategies}

We provide only one strategy which allows expansion of a given node when the
number of visits reaches some user-defined value. For problems where branch
factor of the decision tree is well defined, the expansion threshold should be
set to the mean branch factor of the tree as described in \cite{Pawlewicz}.
Too big values of the expansion threshold lead to small trees and disable add
node selection logic, on the other hand too small values result in a huge tree
which cannot be stored in RAM.

\subsection{Estimation and node selection strategies}

These include (we refer to the node chosen for the next simulation as the most
urgent node and to the node returned after all simulations are performed as the
best node)
\begin{itemize}
  \item \emph{PolicyRandMean} -- chooses node at random when finding the most
    urgent node, the best node is the node with the lowest mean of collected
    estimates
  \item \emph{PolicyEpsMean} -- chooses node at random with probability
    $\epsilon$, otherwise chooses the currently best node as the most urgent
    one, the best node is the one with the lowest mean of collected estimates
  \item \emph{PolicyEpsBest} -- same as above but the best node is the one with the
    lowest minimum of collected estimates
  \item \emph{PolicyMuSigma} -- chooses node with the lowest mean + standard
    deviation of collected estimates, this is the best node too
\end{itemize}

